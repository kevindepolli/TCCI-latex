% ==============================================================================
% TCC - Nome do Aluno
% Capítulo 1 - Introdução
% ==============================================================================
\chapter{Introdução}
\label{sec-intro}

%\hl{Texto.}

%\hrulefill

A \textbf{Introdução} deve conter de \textbf{3 a 5 páginas}. Primeiramente, deve ser colocada a
Descrição do trabalho, a qual apresenta o contexto do trabalho e a definição do escopo do
mesmo. Deve-se delimitar o escopo do trabalho de forma que haja condições
técnicas suficientes para que o mesmo seja concluído em tempo hábil. \cite{albert1999internet}


%%% Início de seção. %%%
\section{O Problema e sua Importância}
\label{sec-intro-probimp}

A \textbf{Motivação} apresenta as circunstancias que interferiram na escolha do tema.
A \textbf{Justificativa} apresenta o porquê da escolha do tema, o problema a ser resolvido
e a relevância do trabalho, referindo-se a estudos anteriores sobre o tema, ressaltando
suas eventuais limitações e destacando a necessidade de se continuar pesquisando o
assunto.


%%% Início de seção. %%%
\section{Objetivos}
\label{sec-intro-obj}

Nesta subseção, deve ser descrito o \textbf{objetivo geral} do trabalho, detalhando em
seguida, seus \textbf{objetivos específicos}.
\subsection{Objetivo geral}
O \textbf{Objetivo Geral} expressa a finalidade do trabalho: para quê? Deve ter coerência
direta com o tema do trabalho e ser apresentado em uma frase que inicie com um verbo
no infinitivo. O objetivo geral do trabalho está relacionado ao resultado principal do trabalho.
\subsection{Objetivos específicos}
Os \textbf{Objetivos Específicos} apresentam os detalhes e/ou desdobramentos do
objetivo geral que levam a resultados intermediários e relevantes para alcançar o objetivo geral. Sempre será mais de um objetivo específico, todos iniciando com verbo no infinitivo.


%OsObjetivos Específicosapresentam os detalhes e/ou desdobramento do objetivogeral. Sempre serão mais de um objetivo, todos iniciando com verbo no infinitivo queapresente tarefas parciais de pesquisa em prol da execução do objetivo geral.
